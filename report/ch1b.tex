\section{Proof using token-based reasoning}
We saw in the previous section how the EFC permission structure is weaker in practice than what our intuition dictates. While this limitation can easily be overcome, it is at the price of slightly less intuitive proofs. We will see here that the token-based reasoning developed in \cite{pascal} is itself weaker than the EFC permission structure it was based on.

If we were to straightforwardly translate the proof developed in the previous section using the tokens defined in \cite{pascal}, we would get the invariant and read and write permissions shown in Figure~\ref{fig:invToksRWFolly}.

\begin{figure}
$$\begin{aligned} \mathcal{Q}^{EFC}(v) :=& \operatorname{let} n=\left\lfloor\frac{v}{4}\right\rfloor + lsb(v), w=\operatorname{lsb}(v), \text { in } \\ & v \geq 0 \wedge (w = 1 ? \texttt{Src}(res, n, write) : \texttt{Src}(res, n, read))
\end{aligned} $$

$$\rPerm = \texttt{Tok}(res, 1, read)$$

$$\wPerm = \texttt{Tok}(res, 1, write)$$
		\caption{Simple invariant and Read-Write permissions for the Folly Reader-Writer Spinlock}
		\label{fig:invToksRWFolly}
\end{figure}

Now, in the \texttt{unlock\_shared} function, we would like to prove that we cannot read $v < 4$ from the location \texttt{bits}. Here we cannot do it, as the token we own is not incompatible with the location invariant. Reading $v < 4$ and $lsb(v) = 1$ from \texttt{bits} yields $\texttt{Src}(res, 1, write)$. Using the \textsc{any-src-read-tok} rule from \cite{pascal}, we could then combine the source with our read token to get $\texttt{Src}(res, 0, read)$, which using the \textsc{all-tokens-a} (from \cite{pascal} as well) yields $\texttt{Src}(res, 0, write)$. We could not reach any contradiction here.

This is because the tokens allow for "downgrading" of a token, from write to read permission. Hence it could happen that a thread gets write permission, downgrades it to read permission, then gives back this permission. The source then has to be able to reconstruct from the number of tokens it has whether or not it now owns full permission or not. This could not happen using the EFC ghost state directly as such a downgrading of permission is not possible: the amount of permission $q$ contained in a ghost location $\ghost{\lambda : (1, q)}$ cannot be made lower without further dividing the token.

We here see that the proof fails earlier than when using the EFC permission structure, showing a small difference between the two reasonings. This can again be easily fixed, by making the proof a little less straightforward. We can for instance use the idea developed in \cite{pascal} for the Rust Atomic Reference Counter proof: defining a different amount of tokens for read an write permissions. This outcome can then be forbidden. By giving 3 tokens for read permission and 2 for write, we would reach a contradiction by trying to combine  $\texttt{Src}(res, 2, write)$ and $\texttt{Tok}(res, 3, read)$, using rule \textsc{too-many-tokens}. This trick allows us to get as far in the proof as we did using the EFC ghost location, but we then get stuck in the next step: proving that $v \geq 4$ and $lsb(v) = 1$ is not possible. This is what we would expect: the tokens are not more powerful than the EFC ghost location.
