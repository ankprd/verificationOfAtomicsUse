We show below a simplification of the code of the glibc \texttt{pthread\_rwlock}. In the following, we use the following abreviations: Acq for acquire, Rel for release and Rlx for relaxed. While a lot of care was taken when transcribing this code, it may still be possible that some errors can still be found, as a lot of simplifications were needed from the original code to this one.

The constant \texttt{WRLOCKED} allows us to directly access the second least significant bit, denoting if there is a thread requesting or having write access, while \texttt{WRPHASE} corresponds to the least significant bit, telling if the lock is in read or write phase.

The number of threads currently having or requesting write access to the lock is \texttt{\_\_readers / 8}, as the third least significant bit is used in the actual implementation. We do not mention it here, as it is only used when using the lock in a particular mode we are not interested in here. 

\begin{lstlisting}
Constants:
WRLOCKED = 2
WRPHASE = 1

int readLock(){
	int r = fetchAndAdd_Acq(__readers, 8) + 8
	if(r & WRPHASE == 0)
		return 0
	while(r & WRPHASE != 0 && r & WRLOCKED == 0){
		if(CAS_Acq(__readers, &r, r ^ WRPHASE))
			return 0;
	}
	//there is a writer (maybe in waiting), it will get us to read mode at some point
	while(Load_Acq(__readers) & WRPHASE == 0){;}
}

int readUnlock(){
	int r = Load_Rlx(__readers)
	while(1){
		int rNew = r - 8
		if(rNew == 0){
			if(rNew & WRLOCKED)
				rNew |= WRPHASE
		}
		if(CAS_Rel(__readers, r, rNew))
			break
	}
}

int writeLock(){
	int r = FetchAndOr_Acq(__readers, WRLOCKED)
	if(r & WRLOCKED){
		while(1){
			if(r & WRLOCKED == 0){
				if(CAS_Acq(__readers, r, r | WRLOCKED))
					break
				continue
			}
			r = Load_Rlx(__readers)
		}
		r |= WRLOCKED
	}
	if(r & WRPHASE)
		return 0
	while(r & WRPHASE == 0 && r / 8 == 0){
		if(CAS_Acq(__readers, r, r | WRPHASE)){
			return 0
		}
	}
	while(Load_Rlx(__readers) & WRPHASE == 0) {;}
	return 0;
}

int writeUnlock(l){
	r = Load_Rlx(__readers)
	while(1){
		int rNew = r ^ WRLOCKED
		if(r / 8 > 0)
			rNew ^= WRPHASE
		if(CAS_Rel(__readers, r, rNew))
			break;
	}
}   
\end{lstlisting}
