\section{Formalizing the specifications}

To produce a formal proof of the properties described above, we need to rewrite again the code of the serial executor, using only language constructs allowed in FSL++. The main change is the use of a \texttt{protected} non atomic variable to model the resources the serial executor protects. Instead of pushing tasks \texttt{func} to the queue, we now push some integer value, and executing a task is simply assigning this value to the \texttt{protected} field. We also replace the \texttt{do ... while} loop with a \texttt{repeat}, that repeats its body until it returns a non-zero value. Finally, the \texttt{drop} function now returns the value it read from \texttt{keepAliveCounter\_}, so that we can have a more precise specification of this function. The resulting simplified code is shown in Figure~\ref{fig:serialExecCP}.

\begin{figure}
	\begin{lstlisting}
Executor newSerialExecutor(Executor parent, Ress protected){
  s = alloc();
  s.protected_ = protected;
  s.parent_ = parent;
  s.queue_ = new Queue();
  store_rel(s.scheduled_, 0);
  store_rlx(s.keepAliveCounter_, 1);
  return s;
}

void add(int f, Executor s){
	s.queue_.enqueue(f);
	s.parent_.add({this->run()});
}

void run(Executor s){
	c = fetch_add_acq(s.scheduled_, 1);
	if(c > 0) {}
	else{
		repeat (
			a = s.queue_.dequeue();
			s.protected_.set(a);
			c = fetch_sub_rel(s.scheduled_, 1);
			(c = 1)//return value of the expression within repeat
		)
	}
}

int drop(Executor s){
	int c = fetch_sub_acq(s.keepAliveCounter_, 1);
	if(c == 1)
		free(s);
	return c;
}

Executor copy(Executor s){
	fetch_add_rlx(s.keepAliveCounter, 1);
	return s;
}
	\end{lstlisting}


	\caption{FSL++ compatible Serial Executor code}
	\label{fig:serialExecCP}
\end{figure}

Verifying our first property, that is to say that the executor guarantees serial execution now amounts to proving that when the write to \texttt{s.protected\_} happens in the \texttt{add} function, we currently have full ownership of the \texttt{s.protected\_} location.

Verifying the second property simply requires showing that when calling \texttt{s.queue\_.dequeue()}, a thread holds the assertion $\consQ$, and when calling \texttt{s.queue\_.enqueue()} it holds $\prodQ$.

Finally, to show that \texttt{keepAliveCounter\_} works as expected, we need to show that there exists a predicate $\mathsf{SE_{\gamma}(s)}$ such that the function specifications shown in Figure~\ref{fig:seFctSpec} hold. This predicate is parameterized by a ghost location we will define in further detail later. For the sake of simplicity, we remove the dependency on the \texttt{protected} location from this predicate.

\begin{figure}
\[
		\{\texttt{protected}^1\} \mathtt{newSerialExecutor(parent, protected)} \{s. \exists \gamma. \mathsf{RE}_\gamma (s)\}
\]

\[
		\{\sePred\} \mathtt{add(f, s)} \{\sePred\}				
\]

\[
		\{\mathsf{U}(\mathtt{s.scheduled\_}, \mathcal{Q})\} \mathtt{run(s)} \{\mathsf{U}(\mathtt{s.scheduled\_}, \mathcal{Q})\}
\]

\[
	\{\sePred\} \mathtt{copy(s)} \{\sePred * \sePred\}
\]

\[
		\{\sePred\} \mathtt{drop(s)} \{y. (y > 1 \land \mathsf{emp}) \lor (y = 1 \land \texttt{s.parent\_}^1) \}
\]

		\caption{Serial executor fonctions specifications}
		\label{fig:seFctSpec}
\end{figure}

Besides, this predicate should allow us to satisfy the two first properties.
